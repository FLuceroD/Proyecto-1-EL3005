% !TEX encoding = UTF-8 Unicode

%------------------------------------
%	PACKAGES AND OTHER DOCUMENT CONFIGURATIONS
%------------------------------------
\documentclass[letterpaper,11pt]{article}
\usepackage[utf8]{inputenc}
\usepackage[T1]{fontenc}
\usepackage{textcomp}
\usepackage{lmodern}
\usepackage[spanish,mexico]{babel} 
\usepackage{graphicx}
\usepackage{subcaption}
\usepackage{float}
\usepackage{blindtext}
\usepackage{fullpage}
\usepackage{amssymb}
\usepackage{mathtools}
\usepackage{algorithm}
\usepackage[noend]{algpseudocode}
\usepackage{nccmath}
\usepackage{enumerate}
\usepackage[none]{hyphenat}
\usepackage{amsmath}
\usepackage{graphicx}
\usepackage{wrapfig}
\usepackage[colorinlistoftodos]{todonotes}
\usepackage[normalem]{ulem}
\useunder{\uline}{\ul}{}
\setlength{\parskip}{2mm}

\begin{document}

\begin{titlepage}

\newcommand{\HRule}{\rule{\linewidth}{0.5mm}} % Defines a new command for the horizontal lines, change thickness here

\center % Center everything on the page
 
%----------------------------------------------------------------------------------------
%	HEADING SECTIONS
%----------------------------------------------------------------------------------------

\textsc{\Large Universidad de Chile, Departamento de Ingeniería Eléctrica}\\[1.5cm] % Name of your university/college
\textsc{\Large Señales y Sistemas I}\\[0.5cm] % Major heading such as course name
\textsc{\large EL-3005-2}\\[0.8cm]
\textsc{\Large Proyecto 1}\\[0.5cm] % Minor heading such as course title

%----------------------------------------------------------------------------------------
%	TITLE SECTION
%----------------------------------------------------------------------------------------

\HRule \\[0.4cm]
{ \huge \bfseries Estudio de Técnicas de Procesamiento de Señales en Sistemas de Sonar}\\[0.1cm] % Title of your document
\HRule \\[1.5cm]
 
%----------------------------------------------------------------------------------------
%	AUTHOR SECTION
%----------------------------------------------------------------------------------------

\begin{minipage}{0.4\textwidth}
\begin{flushleft} \large
\emph{Autor:}\\
Felipe \textsc{Lucero}\\
19.528.232-3 % Your name
\end{flushleft}
\end{minipage}
~
\begin{minipage}{0.4\textwidth}
\begin{flushright} \large
\emph{Profesor:} \\
Jorge F. Silva
\\
[1.0cm]
\emph{Auxiliar:}\\
Roberto Rojas\\
\end{flushright}
\end{minipage}\\[2cm]

% If you don't want a supervisor, uncomment the two lines below and remove the section above
%\Large \emph{Author:}\\
%John \textsc{Smith}\\[3cm] % Your name

%----------------------------------------------------------------------------------------
%	DATE SECTION
%----------------------------------------------------------------------------------------

{\large \today}\\[2cm] 


\vfill 

\end{titlepage}

\newpage
\tableofcontents

\newpage

\section{Descripcion del Problema}


\section{Desarrollo}


\section{Resultados}

\section{Análisis}

\section{Conclusiones}




%\begin{wrapfigure}{r}{0.5\textwidth}
%\vspace{-50pt}
%\begin{center}
%\includegraphics[width=0.5\textwidth]{img/interpol1.png}
%\includegraphics[width=0.5\textwidth]{img/interpol2.png}
%\end{center}
%\vspace{-20pt}
%\caption{Interpolación de una función Lorentziana. Arriba: 5 puntos equiespaciados. Abajo: 12 puntos elegidos al azar}
%\vspace{-100pt}
%\label{interpol1}
%\end{wrapfigure}





%\newpage
%\begin{figure}[H]
%\centering
%\includegraphics[width=1.1\textwidth]{img/lagrange1.png}
%\caption{Interpolación del polinomio de Lagrange para varios puntos equiespaciados}
%\label{dif1}
%\end{figure}
%
%\begin{figure}[H]
%\centering
%\includegraphics[width=1.1\textwidth, height=6.5cm]{img/spline1.png}
%\caption{Interpolación Spline para varios puntos equiespaciados}
%\label{dif2}
%\end{figure}





\end{document}